\documentclass[xcolor=dvipsnames]{beamer}
\mode<presentation>
{
\setbeamercolor{structure}{fg=Red!75!black}
\usetheme{Madrid}
\setbeamercovered{transparent}
}

\usepackage{ucs}
\usepackage[utf8]{inputenc}
\usepackage[czech]{babel}
\usepackage{palatino}
\usepackage{graphicx}
\usepackage{multicol}
\usepackage{verbatim}
\usepackage{url}
\usepackage{array}
\urlstyle{sf}

%%%%%% SET TO CORRECT PATHS
\graphicspath{{/Volumes/dump/pictures/pool/}}

\newcommand{\btVFill}{\vskip0pt plus 1filll}
\newcolumntype{L}[1]{>{\raggedright\let\newline\\\arraybackslash\hspace{0pt}}m{#1}}
\newcolumntype{C}[1]{>{\centering\let\newline\\\arraybackslash\hspace{0pt}}m{#1}}
\newcolumntype{R}[1]{>{\raggedleft\let\newline\\\arraybackslash\hspace{0pt}}m{#1}}

\title[ ]{Introduction to UNIX, vol II}
\author{Kamil S. Jaron}
\date{23.9.~2017}

\usenavigationsymbolstemplate{}
\defbeamertemplate*{footline}{mytheme}{}

\begin{document}

\begin{frame}
	\titlepage
\end{frame}

\begin{frame}
	UNDERSTANDING
\end{frame}

\begin{frame}[fragile]
\large
\begin{verbatim}

$module add UHTS/Quality_control/fastqc/0.11.2

# to check loaded modules
$module list
Currently Loaded Modulefiles:
  1) Development/java_jre/latest
  2) UHTS/Quality_control/fastqc/0.11.2

$fastqc -version
FastQC v0.11.2
\end{verbatim}
\end{frame}

\begin{frame}
	\begin{center}
		\Huge
		bash sees the binary!\\
		\vspace{1cm}
		\Large
		How comes?
	\end{center}
\end{frame}

\begin{frame}[fragile]
\footnotesize
\begin{verbatim}
$type fastqc
fastqc is hashed
(/software/UHTS/Quality_control/fastqc/0.11.2/bin/fastqc)

$echo $PATH
/software/UHTS/Quality_control/fastqc/0.11.2/bin:/software\
/Development/java_jdk/1.8.0_102/jre/bin:/mnt/common/lsf/9.\
1/linux2.6-glibc2.3-x86_64/etc:/mnt/common/lsf/9.1/linux2.\
6-glibc2.3-x86_64/bin:/software/bin:/usr/local/bin:/bin:/u\
sr/bin:/usr/local/sbin:/usr/sbin:/sbin:/home/kjaron/bin
\end{verbatim}
\end{frame}

\begin{frame}[fragile]
\footnotesize
\begin{verbatim}
# remove all modules
$module purge

$module list
No Modulefiles Currently Loaded.

$echo $PATH
/mnt/common/lsf/9.1/linux2.6-glibc2.3-x86_64/etc:/mnt/comm\
on/lsf/9.1/linux2.6-glibc2.3-x86_64/bin:/software/bin:/usr\
/local/bin:/bin:/usr/bin:/usr/local/sbin:/usr/sbin:/sbin:/\
home/kjaron/bin
\end{verbatim}
\end{frame}

\begin{frame}
	\begin{center}
		\Huge
		UNIX organisation \\
		\vspace{1cm}
		\includegraphics[width=0.9\textwidth]{UNIX/filesystem}
	\end{center}
\end{frame}

\begin{frame}[fragile]
	\begin{center}
		\Huge
		Variables
	\end{center}
\huge
\begin{verbatim}
$VAR=5

$echo VAR
VAR

$echo $VAR
5
\end{verbatim}
\end{frame}

\begin{frame}[fragile]
	\begin{center}
		\Huge
		Environmental variables
	\end{center}
\Large
\begin{verbatim}
$echo $USER
kjaron

# terminal prompt
$echo $PS1
\u@\h:\w$

# locations of binaries
$echo $PATH
/usr/local/sbin:/usr/local/bin:
/usr/sbin:/usr/bin:/sbin:/bin

\end{verbatim}
\end{frame}

\begin{frame}[fragile]
	\begin{center}
		\Huge
		Scripts \\
		\Large
		automatisation of your actions
	\end{center}
\end{frame}

\begin{frame}[fragile]
\LARGE
\begin{verbatim}
	$cat who_am_I.sh
	#!/bin/bash

	echo $USER



	$bash who_am_I.sh
	kjaron
\end{verbatim}
\end{frame}

\begin{frame}[fragile]
\LARGE
\begin{verbatim}
	$cat echo.sh
	#!/bin/bash

	echo $1



	$bash echo.sh first second
	first
\end{verbatim}
\end{frame}

\begin{frame}[fragile]
\LARGE
\begin{verbatim}
	$cat parameter_overview.sh
	#!/bin/bash

	echo $#
	echo $1
	echo $2
	echo $*
	echo $0
\end{verbatim}
\end{frame}

\begin{frame}[fragile]
\LARGE
\begin{verbatim}
	$bash parameter_overview.sh \
	> one 2 three 4
	#!/bin/bash

	4
	one
	2
	one 2 three 4
	parameter_overview.sh
\end{verbatim}
\end{frame}

\begin{frame}[fragile]
\LARGE
\begin{verbatim}
	$bash parameter_overview.sh \
	\ one 2 three 4

	$#     3
	$1     one
	$2     2
	$*     one 2 three 4
	$0     parameter_overview.sh
\end{verbatim}
\end{frame}

\begin{frame}[fragile]
	\begin{center}
		\Huge
		Command-line texteditors
	\end{center}
	\Large
\begin{verbatim}
vi <new_name>       # create
vi <exitsing_file>  # edit

# vi is everywhere
# hopefully vim as well

nano <file>         # alternative

# common
# simpler to handle
\end{verbatim}
\end{frame}

\begin{frame}[fragile]
	\begin{center}
		\Large
		\verb!nano count_reads.sh! \\
		\vspace{1cm}
		\includegraphics[width=0.9\textwidth]{UNIX/nano}
	\end{center}
\end{frame}

\begin{frame}[fragile]
	\begin{center}
		\Large
		\verb!vi count_reads.sh! \\
		\vspace{1cm}
		\includegraphics[width=0.9\textwidth]{UNIX/vim_1}
	\end{center}
\end{frame}

\begin{frame}[fragile]
	\begin{center}
		\Large
		\verb!a! or \verb!i! enter insert mode \\
		\vspace{1cm}
		\includegraphics[width=0.9\textwidth]{UNIX/vim_2}
	\end{center}
\end{frame}

\begin{frame}[fragile]
	\begin{center}
		\Large
		\verb!<esc>! escape insert mode\\
		\vspace{1cm}
		\includegraphics[width=0.9\textwidth]{UNIX/vim_3}
	\end{center}
\end{frame}

\begin{frame}[fragile]
	\begin{center}
		\Large
		\verb!:wq! command write quit\\
		\vspace{1cm}
		\includegraphics[width=0.9\textwidth]{UNIX/vim_4}
	\end{center}
\end{frame}

\begin{frame}
	\begin{center}
		\Huge
		Why is that monster popular?\\
		\vspace{1cm}
		\includegraphics[width=0.9\textwidth]{UNIX/vimcheat}
	\end{center}
	\hfill and memes...
\end{frame}

\begin{frame}
	\begin{center}
		\includegraphics[width=0.9\textwidth]{/UNIX/vim_tweet}
	\end{center}
\end{frame}

\begin{frame}[fragile]
\begin{center}
	\Huge
	Your turn
\end{center}
\Large
\begin{verbatim}
vi <file>    # open file
a, i         # insert mode
<esc>        # command mode
:            # command
:w           # write
:wq          # write \& quit
:q           # quit io saving
\end{verbatim}
\end{frame}

\begin{frame}
	\begin{center}
		\Huge
		Hey, would you like to do a first step or a master project with me?!\\
	\end{center}
\end{frame}

\end{document}
