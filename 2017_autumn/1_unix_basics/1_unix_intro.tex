\documentclass[xcolor=dvipsnames]{beamer}
\mode<presentation>
{
\setbeamercolor{structure}{fg=Red!75!black}
\usetheme{Madrid}
\setbeamercovered{transparent}
}

\usepackage{ucs}
\usepackage[utf8]{inputenc}
\usepackage[czech]{babel}
\usepackage{palatino}
\usepackage{graphicx}
\usepackage{multicol}
\usepackage{verbatim}
\usepackage{url}
\usepackage{array}
\urlstyle{sf}

%
% THIS SOURCE FILE IS PUBLIC DOMAIN
% made by Kamil S Jaron with help of internet
% contact: kamiljaron at gmail
%

%%%%%% SET TO CORRECT PATHS
\graphicspath{{/Volumes/dump/pictures/pool/}}

\newcommand{\btVFill}{\vskip0pt plus 1filll}
\newcolumntype{L}[1]{>{\raggedright\let\newline\\\arraybackslash\hspace{0pt}}m{#1}}
\newcolumntype{C}[1]{>{\centering\let\newline\\\arraybackslash\hspace{0pt}}m{#1}}
\newcolumntype{R}[1]{>{\raggedleft\let\newline\\\arraybackslash\hspace{0pt}}m{#1}}

\title[ ]{Introduction to UNIX}
\author{Kamil S Jaron}
\date{21.9.~2017}

\usenavigationsymbolstemplate{}
\defbeamertemplate*{footline}{mytheme}{}

\begin{document}

\Large
\begin{frame}
	\titlepage
\end{frame}

% \begin{frame}
% 	\huge
% 	\begin{center}
% 		\begin{tabular}{c | c | c}
% 		 & today & tomorrow \\
% 		 \hline
% 		AM & navigation & variables \\
% 		 & text processing & scripts \\
% 		 & pipes & cluster \\
% 		 \hline
% 		 PM &  & QC \\
% 		 & & trimming \\
% 		\end{tabular}
% 	\end{center}
% \end{frame}

\section{why unix}

\begin{frame}
	\Huge
	\begin{center}
		Why UNIX?
	\end{center}
\end{frame}

\begin{frame}
	\frametitle{Sequencing this semester}
	\LARGE
	\begin{center}
	Sequencing reads of 11 genomes \\
	~\\
	\pause
	$\approx$ 1GB of letters / genome \\
	~\\
	Each is a book of $\approx$ 370,000 pages
	\end{center}
\end{frame}

\begin{frame}
	\begin{center}
	\huge
	How to open such file?? \\
	\vspace{1cm}

	\Large
	\begin{tabular}{C{0.4\textwidth}C{0.4\textwidth}}
	Notepad? & Office? \\
	\includegraphics[scale=0.3]{UNIX/notepad-icon} & \includegraphics[scale=0.2]{UNIX/office_logo} \\
	\small not well suited for big files & \small Use at least read-only mode\\
	\end{tabular}
	\end{center}
\end{frame}

\begin{frame}
	\begin{center}
		\includegraphics[width=0.9\textwidth]{/UNIX/gene_name_errors_excel}
	\end{center}
\end{frame}


\begin{frame}
	\begin{center}
		\includegraphics[width=0.9\textwidth]{/UNIX/GUI_pipeline}
	\end{center}
\end{frame}


\begin{frame}
	\begin{center}
		\Huge
		Cluster computing\\
		\vspace{1cm}
		\includegraphics[width=0.9\textwidth]{/UNIX/Vital-IT-cluster}
	\end{center}
\end{frame}

\begin{frame}
	\Huge
	\begin{center}
		So what is UNIX?
	\end{center}
	\Large
	\pause
	\begin{itemize}
		\item simple command line environment (bash)
		\item robust, fast and scalable utilities
		\item easy integration of new tools
	\end{itemize}
\end{frame}

\begin{frame}
	\Huge
	\begin{center}
		bash
	\end{center}
	\large
	a command line \textbf{language} for communication with a computer
\end{frame}

\begin{frame}
	\frametitle{How to open terminal}

	\begin{center}
		MyUnil $\rightarrow$ Sequence a Genome I $\rightarrow$ 1\_unix\_basics.pdf
	\end{center}

	\small
	\begin{tabular}{C{0.3\textwidth} | C{0.3\textwidth} | C{0.3\textwidth}}
	 	 \includegraphics[width=0.15\textwidth]{/logos/linux_logo} &
		 \includegraphics[width=0.15\textwidth]{/logos/MacOSX} &
		 \includegraphics[width=0.15\textwidth]{/logos/Microsoft-Windows-Logo} \\
		 ctrl + alt + t &
		 cmd + space \newline type "terminal" \newline Enter &
		 download PuTTY client \newline set hostname to prd.vital-it.ch \newline your vital-it login \newline
		 ~ \newline OR \newline ~ \newline
		 allow "Windows Subsystem for Linux (Beta)" \newline start menu $\rightarrow$ bash\\
	\end{tabular}

\end{frame}

\section{navigation}

\begin{frame}
	\Huge
	\begin{center}
		bash
	\end{center}
	\large
	a command line \textbf{language} for communication with a computer
\end{frame}

\begin{frame}
	bash $\approx$ explorer + toolbox
	\begin{center}
		\includegraphics[width=0.6\textwidth]{/UNIX/Windows-7-Explorer-Tweaked-Folder}
	\end{center}
\end{frame}

\begin{frame}
	\begin{center}
		\includegraphics[width=1\textwidth]{/UNIX/commnad_line}
	\end{center}
\end{frame}


\begin{frame}
	~~~bash \hfill explorer~~~
	\begin{center}
	\begin{tabular}{L{0.5\textwidth}L{0.5\textwidth}}
		\includegraphics[width=0.5\textwidth]{/UNIX/commnad_line} &
		\includegraphics[width=0.47\textwidth]{UNIX/Windows-7-Explorer-Tweaked-Folder} \\
	\end{tabular}
	\end{center}
\end{frame}

\begin{frame}[fragile]
Syntax used in slides
\begin{verbatim}
# comments

command

$command
output1

command <argument>
\end{verbatim}
\end{frame}

\begin{frame}[fragile]
Pay attention where you are!
\begin{verbatim}
# relative paths
.     # this directory
..    # parent directory
~     # my home directory

# absolute path
/     # root directory
      # filename path separator

\end{verbatim}
\end{frame}

\begin{frame}[fragile]
Where is my home?
\begin{verbatim}
# ~
# /home/<USER>
# /home/kamil

kamil@frt:~$ pwd
/home/kamil
\end{verbatim}

\end{frame}


\begin{frame}[fragile]
	\begin{verbatim}
		# mind the space between cmd and arg
		command <arguments>
	\end{verbatim}
	Navigation commands:
	\begin{verbatim}
		pwd        # print working directory
		ls <where>        # list segments
		cd <to>           # change directory
	\end{verbatim}
\end{frame}

\begin{frame}
	\Huge
	\begin{center}
		demonstration
	\end{center}
\end{frame}

\begin{frame}[fragile]
Quiz!
\begin{verbatim}
kamil@frt:~$ ls
kamil@frt:~$ ls .
kamil@frt:~$ ls ../kamil
kamil@frt:~$ ls ~
kamil@frt:~$ ls /home/kamil
\end{verbatim}
Which of the commands will list files in my home?
\end{frame}

\section{explorer}

\begin{frame}[fragile]
\begin{verbatim}
	# more operations in explorer
	mkdir       # make directory
	rm          # remove file

	cp <what> <where> # copy
	mv <what> <where> # move

	man <command> # manual
\end{verbatim}
\end{frame}

\begin{frame}[fragile]
	\begin{verbatim}
		command -<parameters> <arguments>
	\end{verbatim}
	Examples:
	\begin{verbatim}
		ls -lah  #list long, all, human readable
		ls -la ..  #list in parent directory
		cp -r <what> <were> # recursive
		rm -rf <what>       # -||- , force

		man <command>
	\end{verbatim}
\end{frame}

\begin{frame}[fragile]
	\Large
	\begin{verbatim}
		# Manual
		man <command>


		# Auto Completion by <tab>
		cd /<tab><tab>   # lists all in root
		cd ~/<tab><tab>  # lists all in home



	\end{verbatim}
\end{frame}

\begin{frame}[fragile]
	\Large
	\begin{verbatim}
		# Manual
		man <command>

		# Auto Completion by <tab>
		cd /<tab><tab>   # lists all in root
		cd ~/<tab><tab>  # lists all in home

		# Command history
		<arrow_up>       # last excuted command
		<Ctrl+R>         # full-text search
	\end{verbatim}
\end{frame}

\begin{frame}
	\Huge
	\begin{center}
		demonstration
	\end{center}
\end{frame}

\begin{frame}[fragile]
	\begin{center}
		MyUnil $\rightarrow$ Sequence a Genome I $\rightarrow$ 1\_unix\_basics.pdf
	\end{center}
	\Large
	\begin{verbatim}
		# Manual
		man <command>

		# Auto Completion by <tab>
		cd /<tab><tab>   # lists all in root
		cd ~/<tab><tab>  # lists all in home

		# Command history
		<arrow_up>       # last excuted command
		<Ctrl+R>         # full-text search
	\end{verbatim}
\end{frame}

\begin{frame}
	\Huge
	\begin{center}
		break
	\end{center}
\end{frame}

\begin{frame}
	\Huge
	\begin{center}
		bash
	\end{center}
	\large
	a command line \huge \textbf{language} \large for communication with a computer
\end{frame}

\begin{frame}[fragile]
	\frametitle{Special bash symbols}
	\large
	\begin{tabular}{L{0.45\textwidth} L{0.5\textwidth}}
	Comments & \verb!#! \\
	Navigation symbols & \verb!. .. ~ /! \\
	Separator of arguments & \verb!<space>! \\
	Command separator & \verb!<newline> ;! \\
	& \\
	Expansion characters & \verb!* ? [] {}! \\
	Assign variable & \verb!=! \\
	Call variable & \verb!$! \\
	& \\
	Command substitution & \verb!$(<cmd>) `<cmd>`! \\
	Redirection & \verb!> | < &! \\
	& \\
	Literal strings & \verb!'' ""! \\
	Escape character & \verb!\! \\
	\end{tabular}
\end{frame}

\section{text operations}

\begin{frame}
	\Huge
	\begin{center}
		What about that toolbox?
	\end{center}
\end{frame}

\begin{frame}
	\begin{center}
	\huge
	How to check a file?? \\
	\vspace{1cm}

	\Large
	\begin{tabular}{C{0.4\textwidth}C{0.4\textwidth}}
	Notepad? & Office? \\
	\includegraphics[scale=0.3]{UNIX/notepad-icon} & \includegraphics[scale=0.2]{UNIX/office_logo} \\
	\small not well suited for big files & \small Use at least read-only mode\\
	\end{tabular}
	\end{center}
\end{frame}

\begin{frame}[fragile]
	\huge
	Look at a plain-text file
	\Large
	\begin{verbatim}
		command <text_file>

		echo
		head / tail
		cat   # catenate
		less  # > more; text reader

		wc    # word count
		tr    # translate
		grep  # global regular expression print
	\end{verbatim}
\end{frame}

\begin{frame}[fragile]
	\huge
	IO streams - a way how to build pipes!
	\Large
	\begin{verbatim}
		command <text_file>
	           == cat <text_file> | command
		echo
		tail / head
		cat   # catenate
		less  # > more

		wc    # word count
		tr    # translate
		grep  # global regular expression print
	\end{verbatim}
\end{frame}

\begin{frame}
	\begin{center}
		\includegraphics[width=0.9\textwidth]{/UNIX/GUI_pipeline}
	\end{center}
\end{frame}

\begin{frame}[fragile]
	\huge
	IO streams!
	\Large
	\begin{verbatim}


		ls | wc -w
	\end{verbatim}
\end{frame}

\begin{frame}[fragile]
	\huge
	Streams in the service of genomics!
	\Large
	\begin{verbatim}


		grep ">" seq.fasta | wc -l



				>seq1
				CGATCGTCGTAGCTACGAT
				>seq2
				ACCGATCAAACCGTCGTAA
	\end{verbatim}
\end{frame}

\begin{frame}[fragile]
	\huge
	input stream (file)
	\Large
	\begin{verbatim}


		grep -v ">" seq.fasta | \
		  tr -d '\n' | wc -c



		>seq1
		CGATCGTCGTAGCTACGAT
		>seq2
		ACCGATCAAACCGTCGTAA
	\end{verbatim}
\end{frame}

\begin{frame}[fragile]
	\huge
	\verb!grep -v ">"!
	\Large
	\begin{verbatim}


		grep -v ">" seq.fasta



		CGATCGTCGTAGCTACGAT
		ACCGATCAAACCGTCGTAA
	\end{verbatim}
\end{frame}

\begin{frame}[fragile]
	\huge
	\verb!tr -d "\n"!
	\Large
	\begin{verbatim}


		grep -v ">" seq.fasta | \
		  tr -d "\n"


		CGATCGTCGTAGCTACGATACCGATCAAACCGTCGTAA
	\end{verbatim}
\end{frame}

\begin{frame}[fragile]
	\huge
	\verb!wc -c!
	\Large
	\begin{verbatim}


		grep -v ">" seq.fasta | \
		  tr -d "\n" | wc -c


		38
	\end{verbatim}
\end{frame}

\begin{frame}[fragile]
	\huge
	Stream redirection overview
	\Large
	\begin{verbatim}


		grep ">" seq.fasta         # print
		grep ">" seq.fasta | wc    # pipe
		grep ">" seq.fasta > file  # write
		grep ">" seq.fasta >> file # append
	\end{verbatim}
\end{frame}

\begin{frame}[fragile]
	\huge
	Your turn
	\Large
	\begin{verbatim}
		command <parameters> <text_file>

		echo  # print text
		head / tail # start/end of file
		cat   # print file(s)
		less  # text-file browser

		wc    # word count
		tr    # translate
		grep  # print lines with pattern
	\end{verbatim}
\end{frame}

\section{recap}

\begin{frame}[fragile]
	$\sum$ relative and absolute path
	\Large
	\begin{verbatim}
		# relative
		./my_dir/project1/data
		~/photos
		../others/
		# absolute
		/dev/null
		/home/kamil
	\end{verbatim}
	Every command is executed somewhere!
\end{frame}

\begin{frame}[fragile]
	$\sum$ efficient control
	\Large
	\begin{verbatim}
		# Manual
		man <command>

		# Auto Completion by <tab>
		cd /<tab><tab>   # lists all in root
		cd ~/<tab><tab>  # lists all in home

		# Command history
		<arrow_up>       # last excuted command
		<Ctrl+R>         # full-text search
	\end{verbatim}
\end{frame}

\begin{frame}[fragile]
	$\sum$ navigation
	\Large
	\begin{verbatim}
		pwd                # print workding dir
		ls                 # list segments
		cd                 # change directory
		mkdir              # make directory
		rm                 # remove file
		cp <what> <where>  # copy
		mv <what> <where>  # move

		man <command>      # manual
	\end{verbatim}
\end{frame}

\begin{frame}[fragile]
	$\sum$ text operations
	\Large
	\begin{verbatim}
		ls | wc            # piping
		ls > my_segments   # redirection
		pwd >> my_segments # append

		echo          # print text
		head / tail   # start/end of file
		cat           # print file(s)
		less          # text-file browser

		wc     # word count
		tr     # translate
		grep   # print lines with pattern
	\end{verbatim}
\end{frame}

\begin{frame}[fragile]
	\begin{center}
		\Huge
		See you tomorrow!
	\end{center}
\end{frame}

\end{document}
