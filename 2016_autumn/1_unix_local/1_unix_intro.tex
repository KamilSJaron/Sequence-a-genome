\documentclass[xcolor=dvipsnames]{beamer} 
\mode<presentation>
{
\setbeamercolor{structure}{fg=Red!75!black}
\usetheme{Madrid} 
\setbeamercovered{transparent}
}

\usepackage{ucs}
\usepackage[utf8]{inputenc}
\usepackage[czech]{babel}
\usepackage{palatino}
\usepackage{graphicx}
\usepackage{multicol}
\usepackage{verbatim}
\usepackage{url}
\usepackage{array}
\urlstyle{sf}

%%%%%% SET TO CORRECT PATHS
\graphicspath{{/Volumes/dump/pictures/pool/}}

\newcommand{\btVFill}{\vskip0pt plus 1filll}
\newcolumntype{L}[1]{>{\raggedright\let\newline\\\arraybackslash\hspace{0pt}}m{#1}}
\newcolumntype{C}[1]{>{\centering\let\newline\\\arraybackslash\hspace{0pt}}m{#1}}
\newcolumntype{R}[1]{>{\raggedleft\let\newline\\\arraybackslash\hspace{0pt}}m{#1}}

\title[ ]{Introduction to UNIX}
\author{Kamil S Jaron, Marc Robinson-Rechavi}
\date{22.9.~2016}

\usenavigationsymbolstemplate{}
\defbeamertemplate*{footline}{mytheme}{}

\begin{document}

\Large
\begin{frame}
	\titlepage
\end{frame}

\begin{frame}
	\huge
	\begin{center}
	Sequencing reads of 14 genomes \\
	~\\
	$\approx$ 1GB of plain text data / species \\
	~\\
	$\approx$ 640938 of pages
	\end{center}
\end{frame}

\begin{frame}
	\Huge
	\begin{center}
		Why UNIX?	
	\end{center}
\end{frame}

\begin{frame}
	\begin{center}
	\huge
	How to check a file?? \\
	\vspace{1cm}
	
	\Large
	\begin{tabular}{C{0.4\textwidth}C{0.4\textwidth}}
	Notepad? & Office? \\
	\includegraphics[scale=0.3]{UNIX/notepad-icon} & \includegraphics[scale=0.2]{UNIX/office_logo} \\
	\small not well suited for big files & \small Use at least read-only mode\\
	\end{tabular}
	\end{center}
\end{frame}

\begin{frame}
	\begin{center}
		\includegraphics[width=0.9\textwidth]{/UNIX/gene_name_errors_excel}
	\end{center}
\end{frame}


\begin{frame}
	\begin{center}
		\includegraphics[width=0.9\textwidth]{/UNIX/GUI_pipeline}
	\end{center}
\end{frame}


\begin{frame}
	\begin{center}
		\Huge
		Cluster computing\\
		\vspace{1cm}
		\includegraphics[width=0.9\textwidth]{/UNIX/Vital-IT-cluster}
	\end{center}
\end{frame}	

\begin{frame}
	\Huge
	\begin{center}
		Well, and how can UNIX help us?	
	\end{center}
\end{frame}

\begin{frame}
	Command line $\approx$ explorer + toolbox of commands
	\begin{center}
		\includegraphics[width=0.6\textwidth]{/UNIX/Windows-7-Explorer-Tweaked-Folder}
	\end{center}
\end{frame}

\begin{frame}
	bash $\approx$ explorer
	\begin{center}
		\includegraphics[width=0.6\textwidth]{/UNIX/Windows-7-Explorer-Tweaked-Folder}
	\end{center}
\end{frame}

\begin{frame}
	\begin{center}
		\includegraphics[width=1\textwidth]{/UNIX/commnad_line}
	\end{center}
\end{frame}

\begin{frame}
	bash $\approx$ explorer (where am I?)
	\begin{center}
		\includegraphics[width=1.8\textwidth]{/UNIX/pwd}
	\end{center}
\end{frame}

\begin{frame}
	bash $\approx$ explorer (browse directories)
	\begin{center}
		\includegraphics[width=1.4\textwidth]{/UNIX/cd_ls}
	\end{center}
\end{frame}

\begin{frame}
	bash $\approx$ explorer (copy and move files)
	\begin{center}
		\includegraphics[width=1\textwidth]{/UNIX/relative_path_cp}
	\end{center}
\end{frame}

\begin{frame}[fragile]
\begin{verbatim}
### relative paths
.     # this directory
..    # parent directory
~     # my home directory
### absolute path
/     # root directory
\end{verbatim}
\end{frame}

\begin{frame}
	bash $\approx$ explorer (remove files)
	\begin{center}
		\includegraphics[width=1\textwidth]{/UNIX/mv_rm}
	\end{center}
\end{frame}

\begin{frame}[fragile]
	\begin{verbatim}
		command -<parameters> <arguments>
	\end{verbatim}Í
	Examples:
	\begin{verbatim}
		ls -lah  #list long, all, human readable
		ls -la ..  #list in parent directory
		cp -r <what_dir> <were> # recursive
		rm -rf <what dir>       # -||- , force
		# careful with this one...
	\end{verbatim}
\end{frame}

\begin{frame}[fragile]
	\Huge
	\begin{center}
		OK, try it!	
	\end{center}
	\Large
	\begin{verbatim}
		
		
		
		
		
		
	\end{verbatim}
\end{frame}

\begin{frame}[fragile]
	\Huge
	\begin{center}
		OK, try it!	
	\end{center}
	\Large
	\begin{verbatim}
		# Auto Completion by <tab>
		cd /<tab><tab>   # lists all in root
		cd ~/k<tab><tab> # lists all in home
		
		
		
	\end{verbatim}
\end{frame}

\begin{frame}[fragile]
	\Huge
	\begin{center}
		OK, try it!	
	\end{center}
	\Large
	\begin{verbatim}
		# Auto Completion by <tab>
		cd /<tab><tab>   # lists all in root
		cd ~/k<tab><tab> # lists all in home
		# Command history
		<arrow_up>       # last excuted command
		<Ctrl+R>         # full-text search
	\end{verbatim}
\end{frame}

\begin{frame}
	\Huge
	\begin{center}
		What about that toolbox?	
	\end{center}
\end{frame}

\end{document}



